% Options for packages loaded elsewhere
\PassOptionsToPackage{unicode}{hyperref}
\PassOptionsToPackage{hyphens}{url}
\documentclass[
]{article}
\usepackage{xcolor}
\usepackage[margin=1in]{geometry}
\usepackage{amsmath,amssymb}
\setcounter{secnumdepth}{-\maxdimen} % remove section numbering
\usepackage{iftex}
\ifPDFTeX
  \usepackage[T1]{fontenc}
  \usepackage[utf8]{inputenc}
  \usepackage{textcomp} % provide euro and other symbols
\else % if luatex or xetex
  \usepackage{unicode-math} % this also loads fontspec
  \defaultfontfeatures{Scale=MatchLowercase}
  \defaultfontfeatures[\rmfamily]{Ligatures=TeX,Scale=1}
\fi
\usepackage{lmodern}
\ifPDFTeX\else
  % xetex/luatex font selection
\fi
% Use upquote if available, for straight quotes in verbatim environments
\IfFileExists{upquote.sty}{\usepackage{upquote}}{}
\IfFileExists{microtype.sty}{% use microtype if available
  \usepackage[]{microtype}
  \UseMicrotypeSet[protrusion]{basicmath} % disable protrusion for tt fonts
}{}
\makeatletter
\@ifundefined{KOMAClassName}{% if non-KOMA class
  \IfFileExists{parskip.sty}{%
    \usepackage{parskip}
  }{% else
    \setlength{\parindent}{0pt}
    \setlength{\parskip}{6pt plus 2pt minus 1pt}}
}{% if KOMA class
  \KOMAoptions{parskip=half}}
\makeatother
\usepackage{graphicx}
\makeatletter
\newsavebox\pandoc@box
\newcommand*\pandocbounded[1]{% scales image to fit in text height/width
  \sbox\pandoc@box{#1}%
  \Gscale@div\@tempa{\textheight}{\dimexpr\ht\pandoc@box+\dp\pandoc@box\relax}%
  \Gscale@div\@tempb{\linewidth}{\wd\pandoc@box}%
  \ifdim\@tempb\p@<\@tempa\p@\let\@tempa\@tempb\fi% select the smaller of both
  \ifdim\@tempa\p@<\p@\scalebox{\@tempa}{\usebox\pandoc@box}%
  \else\usebox{\pandoc@box}%
  \fi%
}
% Set default figure placement to htbp
\def\fps@figure{htbp}
\makeatother
\setlength{\emergencystretch}{3em} % prevent overfull lines
\providecommand{\tightlist}{%
  \setlength{\itemsep}{0pt}\setlength{\parskip}{0pt}}
\usepackage{bookmark}
\IfFileExists{xurl.sty}{\usepackage{xurl}}{} % add URL line breaks if available
\urlstyle{same}
\hypersetup{
  pdftitle={Trends in NFL Player Representation from SEC Schools (2011--2025)},
  pdfauthor={Omar Sepulveda, Riccardo Gutierrez, Naomi Menard},
  hidelinks,
  pdfcreator={LaTeX via pandoc}}

\title{Trends in NFL Player Representation from SEC Schools
(2011--2025)}
\author{Omar Sepulveda, Riccardo Gutierrez, Naomi Menard}
\date{2025-12-05}

\begin{document}
\maketitle

\section{Abstract}\label{abstract}

The NCAA SEC is one of the premier conferences in college football from
recent years, but how much is each program represented in the draft? Our
data was collected from Pro-Football-Reference.com, which is a database
that tracks every NFL player's stats, awards, drafting, and teams during
college and NFL. Our analysis examines the number of draftees from each
SEC football program for every individual year from 2011 through 2025
and compares trends within the graphs to further analyze the popularity
and production of each program (in terms of outputting NFL talent). The
results yield two common patterns in the data: either the program shows
a quadratic trend that peaks in the late 2010s, or an exponential trend
in which draft representation steadily grows from 2011 onward.

\section{Introduction}\label{introduction}

The NFL is the highest viewed sport in the U.S. as it has been nicknamed
Americas sport by the media. In turn college football also exudes major
viewer retention and support since the players in college are the future
of the NFL, also college students love to support their school. Over the
last 15 years there has been one NCAA football conference that has
prominently outputted the most NFL talent, that being the South Eastern
Conference (SEC). Numerous factors apply when it comes to recruitment,
such as head coach prestige, academic prestige, and program prestige,
all these factors apply but the most important factor is players
drafted. Powerhouses like Alabama and Georgia have remained constant
machines when fostering NFL talent, but over recent years LSU and Texas
have provided incredible NFL talent. From 2011 to the present year 2025
the SEC has remained a constant pool of NFL talent as they are the
programs that every high school football player strives to be apart and
compete with the talent within the SEC. The trends for each SEC program
in terms of draft representation will differ considering that each
program competes against all the other programs within the SEC,
therefore more recruits will strive to attend the programs that are
winning and the programs that create the most NFL talent. Therefore, if
we can analyze the trends for each SEC program in terms of draft
representation we can comprehend how each program performed during the
period 2011-2025 and which programs grew or withered in draft
representation.

\section{Data}\label{data}

The data set used for this analysis, ``College\_Dataset.xlsx,'' contains
1470 observations of information on NFL players and the colleges they
attended, with players who began their NFL careers from 2011 to 2025,
spanning 14 years. The data was obtained from Pro-Football-Reference, a
reliable source that collects and publishes official NFL statistics,
including player backgrounds and school affiliations. Relevant variables
include each player's name, the college they attended, and the range of
years they played in the NFL. During the cleaning process, unnecessary
columns such as AP1, PB, St, wAV, Ht, and Wt were removed, keeping only
the relevant variables, while still keeping the 1470 observations in
tact. Additionally, since the data set included players from earlier
eras, we extracted the starting year from each player's career range and
filtered the data to include only those who began their NFL careers in
2011 or later. The observations are not evenly distributed across the
years, with the number of new players varying annually based on factors
such as un-drafted signings. The data includes players from 16 unique
schools. The data provides the individual starting year for each player,
as the yearly data enables more granular and insightful analysis. The
data is entirely real and will not be generated through any randomized
simulation.

\section{Visualization \& Analysis}\label{visualization-analysis}

\includegraphics[trim=1cm]{FinalReport_files/figure-latex/unnamed-chunk-3-1}
\includegraphics[trim=1cm]{FinalReport_files/figure-latex/unnamed-chunk-3-2}
\includegraphics[trim=1cm]{FinalReport_files/figure-latex/unnamed-chunk-3-3}
\includegraphics[trim=1cm]{FinalReport_files/figure-latex/unnamed-chunk-3-4}

Here we have displayed individual bar charts for each SEC school,
showing the number of players that were selected in the NFL Draft in
each individual year from 2011 through 2025. Based on first impressions,
without considering the exact numeric counts, there are clear trends
indicating which programs increased or decreased in draft representation
over time. Georgia, Florida, Alabama, South Carolina, Ole Miss, and
Texas have shown significant growth in draft representation, while many
of the other SEC programs have displayed either steady or declining
results across the years.

\begin{verbatim}
## # A tibble: 237 x 3
##     School         Start Players
##     <chr>          <int>   <int>
##   1 Alabama         2011       6
##   2 Alabama         2012       7
##   3 Alabama         2013      11
##   4 Alabama         2014       7
##   5 Alabama         2015       9
##   6 Alabama         2016       7
##   7 Alabama         2017      13
##   8 Alabama         2018      15
##   9 Alabama         2019      13
##  10 Alabama         2020      14
##  11 Alabama         2021      12
##  12 Alabama         2022       8
##  13 Alabama         2023      14
##  14 Alabama         2024      13
##  15 Alabama         2025      12
##  16 Arkansas        2011       1
##  17 Arkansas        2012       7
##  18 Arkansas        2013       5
##  19 Arkansas        2014       4
##  20 Arkansas        2015       6
##  21 Arkansas        2016       6
##  22 Arkansas        2017       6
##  23 Arkansas        2018       4
##  24 Arkansas        2019       4
##  25 Arkansas        2020       5
##  26 Arkansas        2021       4
##  27 Arkansas        2022       4
##  28 Arkansas        2023       2
##  29 Arkansas        2024       5
##  30 Arkansas        2025       3
##  31 Auburn          2011       8
##  32 Auburn          2012       2
##  33 Auburn          2013       3
##  34 Auburn          2014       7
##  35 Auburn          2015      10
##  36 Auburn          2016       9
##  37 Auburn          2017       5
##  38 Auburn          2018       8
##  39 Auburn          2019       7
##  40 Auburn          2020       7
##  41 Auburn          2021       7
##  42 Auburn          2022       3
##  43 Auburn          2023       7
##  44 Auburn          2024       5
##  45 Auburn          2025       6
##  46 Florida         2011       8
##  47 Florida         2012       5
##  48 Florida         2013      10
##  49 Florida         2014      11
##  50 Florida         2015      11
##  51 Florida         2016      13
##  52 Florida         2017      10
##  53 Florida         2018       7
##  54 Florida         2019       8
##  55 Florida         2020      10
##  56 Florida         2021      11
##  57 Florida         2022       7
##  58 Florida         2023       7
##  59 Florida         2024       6
##  60 Florida         2025      12
##  61 Georgia         2011      10
##  62 Georgia         2012       7
##  63 Georgia         2013      10
##  64 Georgia         2014       1
##  65 Georgia         2015       8
##  66 Georgia         2016       6
##  67 Georgia         2017       3
##  68 Georgia         2018       6
##  69 Georgia         2019       7
##  70 Georgia         2020      16
##  71 Georgia         2021      13
##  72 Georgia         2022      16
##  73 Georgia         2023      16
##  74 Georgia         2024      12
##  75 Georgia         2025      17
##  76 Kentucky        2011       1
##  77 Kentucky        2012       3
##  78 Kentucky        2013       1
##  79 Kentucky        2014       2
##  80 Kentucky        2015       3
##  81 Kentucky        2016       1
##  82 Kentucky        2019       4
##  83 Kentucky        2020       4
##  84 Kentucky        2021       7
##  85 Kentucky        2022       8
##  86 Kentucky        2023       3
##  87 Kentucky        2024       6
##  88 Kentucky        2025       3
##  89 LSU             2011       8
##  90 LSU             2012       6
##  91 LSU             2013       9
##  92 LSU             2014      13
##  93 LSU             2015       7
##  94 LSU             2016       6
##  95 LSU             2017      11
##  96 LSU             2018      10
##  97 LSU             2019       6
##  98 LSU             2020      15
##  99 LSU             2021      11
## 100 LSU             2022       8
## 101 LSU             2023      12
## 102 LSU             2024      10
## 103 LSU             2025       6
## 104 Mississippi St  2011       4
## 105 Mississippi St  2012       3
## 106 Mississippi St  2013       4
## 107 Mississippi St  2014       3
## 108 Mississippi St  2015       5
## 109 Mississippi St  2016       3
## 110 Mississippi St  2018       7
## 111 Mississippi St  2019       6
## 112 Mississippi St  2020       8
## 113 Mississippi St  2021       2
## 114 Mississippi St  2022       5
## 115 Mississippi St  2023       4
## 116 Mississippi St  2024       5
## 117 Mississippi St  2025       1
## 118 Missouri        2011       5
## 119 Missouri        2012       4
## 120 Missouri        2013       2
## 121 Missouri        2014       4
## 122 Missouri        2015       5
## 123 Missouri        2016       3
## 124 Missouri        2017       4
## 125 Missouri        2018       1
## 126 Missouri        2019       2
## 127 Missouri        2020       7
## 128 Missouri        2021       5
## 129 Missouri        2022       4
## 130 Missouri        2023       1
## 131 Missouri        2024       9
## 132 Missouri        2025       4
## 133 Oklahoma        2011       5
## 134 Oklahoma        2012       7
## 135 Oklahoma        2013       4
## 136 Oklahoma        2014       9
## 137 Oklahoma        2015       9
## 138 Oklahoma        2016       3
## 139 Oklahoma        2017       6
## 140 Oklahoma        2018       4
## 141 Oklahoma        2019      10
## 142 Oklahoma        2020       7
## 143 Oklahoma        2021       6
## 144 Oklahoma        2022       7
## 145 Oklahoma        2023       6
## 146 Oklahoma        2024       9
## 147 Oklahoma        2025       3
## 148 Ole Miss        2011       5
## 149 Ole Miss        2012       3
## 150 Ole Miss        2013       1
## 151 Ole Miss        2014       1
## 152 Ole Miss        2015       1
## 153 Ole Miss        2016       6
## 154 Ole Miss        2017       6
## 155 Ole Miss        2018       6
## 156 Ole Miss        2019       6
## 157 Ole Miss        2020       6
## 158 Ole Miss        2021       3
## 159 Ole Miss        2022       8
## 160 Ole Miss        2023       9
## 161 Ole Miss        2024       6
## 162 Ole Miss        2025      12
## 163 South Carolina  2011       4
## 164 South Carolina  2012       8
## 165 South Carolina  2013       7
## 166 South Carolina  2014       5
## 167 South Carolina  2015       3
## 168 South Carolina  2016       5
## 169 South Carolina  2017       2
## 170 South Carolina  2018       4
## 171 South Carolina  2019       5
## 172 South Carolina  2020       7
## 173 South Carolina  2021       7
## 174 South Carolina  2022       5
## 175 South Carolina  2023       7
## 176 South Carolina  2024       7
## 177 South Carolina  2025      13
## 178 Tennessee       2011       5
## 179 Tennessee       2012       1
## 180 Tennessee       2013       5
## 181 Tennessee       2014       6
## 182 Tennessee       2015       3
## 183 Tennessee       2016       3
## 184 Tennessee       2017       8
## 185 Tennessee       2018       7
## 186 Tennessee       2019       3
## 187 Tennessee       2020       3
## 188 Tennessee       2021       7
## 189 Tennessee       2022       9
## 190 Tennessee       2023       8
## 191 Tennessee       2024       9
## 192 Tennessee       2025       7
## 193 Texas           2011       7
## 194 Texas           2012       4
## 195 Texas           2013       5
## 196 Texas           2014       4
## 197 Texas           2015       5
## 198 Texas           2016       3
## 199 Texas           2017       5
## 200 Texas           2018       6
## 201 Texas           2019       5
## 202 Texas           2020       7
## 203 Texas           2021       6
## 204 Texas           2022       5
## 205 Texas           2023       5
## 206 Texas           2024      13
## 207 Texas           2025      12
## 208 Texas A&M       2011       1
## 209 Texas A&M       2012       7
## 210 Texas A&M       2013       4
## 211 Texas A&M       2014       7
## 212 Texas A&M       2015       4
## 213 Texas A&M       2016       4
## 214 Texas A&M       2017       6
## 215 Texas A&M       2018       6
## 216 Texas A&M       2019       9
## 217 Texas A&M       2020       5
## 218 Texas A&M       2021       6
## 219 Texas A&M       2022       6
## 220 Texas A&M       2023       6
## 221 Texas A&M       2024       7
## 222 Texas A&M       2025       6
## 223 Vanderbilt      2011       1
## 224 Vanderbilt      2012       3
## 225 Vanderbilt      2013       1
## 226 Vanderbilt      2014       4
## 227 Vanderbilt      2015       2
## 228 Vanderbilt      2016       2
## 229 Vanderbilt      2017       3
## 230 Vanderbilt      2018       4
## 231 Vanderbilt      2019       4
## 232 Vanderbilt      2020       2
## 233 Vanderbilt      2021       3
## 234 Vanderbilt      2022       1
## 235 Vanderbilt      2023       1
## 236 Vanderbilt      2024       6
## 237 Vanderbilt      2025       1
\end{verbatim}

The numeric table provides a detailed yearly breakdown of the number of
drafted players from each SEC school between 2011 and 2025. Each row
represents a school, while the ``Start'' column corresponds to a
specific draft year. The ``Players'' values indicate exactly how many
players that program contributed to the NFL Draft in that given year.
Years with no drafted players are represented as zeros, ensuring a
complete and consistent comparison across all schools. This table
presents the raw counts directly, making it useful for understanding
absolute draft contributions and for comparing overall volume across
programs over the entire 15-year span.

\begin{verbatim}
## # A tibble: 16 x 15
## # Groups:   School [16]
##    School         `2011→2012` `2012→2013` `2013→2014` `2014→2015` `2015→2016`
##    <chr>                <dbl>       <dbl>       <dbl>       <dbl>       <dbl>
##  1 Alabama               16.7        57.1       -36.4        28.6       -22.2
##  2 Arkansas             600         -28.6       -20          50           0  
##  3 Auburn               -75          50         133.         42.9       -10  
##  4 Florida              -37.5       100          10           0          18.2
##  5 Georgia              -30          42.9       -90         700         -25  
##  6 Kentucky             200         -66.7       100          50         -66.7
##  7 LSU                  -25          50          44.4       -46.2       -14.3
##  8 Mississippi St       -25          33.3       -25          66.7       -40  
##  9 Missouri             -20         -50         100          25         -40  
## 10 Oklahoma              40         -42.9       125           0         -66.7
## 11 Ole Miss             -40         -66.7         0           0         500  
## 12 South Carolina       100         -12.5       -28.6       -40          66.7
## 13 Tennessee            -80         400          20         -50           0  
## 14 Texas                -42.9        25         -20          25         -40  
## 15 Texas A&M            600         -42.9        75         -42.9         0  
## 16 Vanderbilt           200         -66.7       300         -50           0  
##    `2016→2017` `2017→2018` `2018→2019` `2019→2020` `2020→2021` `2021→2022`
##          <dbl>       <dbl>       <dbl>       <dbl>       <dbl>       <dbl>
##  1        85.7       15.4        -13.3        7.69       -14.3       -33.3
##  2         0        -33.3          0         25          -20           0  
##  3       -44.4       60          -12.5        0            0         -57.1
##  4       -23.1      -30           14.3       25           10         -36.4
##  5       -50        100           16.7      129.         -18.8        23.1
##  6        NA         NA           NA          0           75          14.3
##  7        83.3       -9.09       -40        150          -26.7       -27.3
##  8        NA         NA          -14.3       33.3        -75         150  
##  9        33.3      -75          100        250          -28.6       -20  
## 10       100        -33.3        150        -30          -14.3        16.7
## 11         0          0            0          0          -50         167. 
## 12       -60        100           25         40            0         -28.6
## 13       167.       -12.5        -57.1        0          133.         28.6
## 14        66.7       20          -16.7       40          -14.3       -16.7
## 15        50          0           50        -44.4         20           0  
## 16        50         33.3          0        -50           50         -66.7
##    `2022→2023` `2023→2024` `2024→2025`
##          <dbl>       <dbl>       <dbl>
##  1        75         -7.14       -7.69
##  2       -50        150         -40   
##  3       133.       -28.6        20   
##  4         0        -14.3       100   
##  5         0        -25          41.7 
##  6       -62.5      100         -50   
##  7        50        -16.7       -40   
##  8       -20         25         -80   
##  9       -75        800         -55.6 
## 10       -14.3       50         -66.7 
## 11        12.5      -33.3       100   
## 12        40          0          85.7 
## 13       -11.1       12.5       -22.2 
## 14         0        160          -7.69
## 15         0         16.7       -14.3 
## 16         0        500         -83.3
\end{verbatim}

\includegraphics[trim=1cm]{FinalReport_files/figure-latex/unnamed-chunk-6-1}
The heat map displays the year-to-year percentage change in NFL draft
selections for each SEC program from 2011 to 2025. Each tile represents
how a school's draft output changed compared to the previous year. Blue
shades indicate positive growth in drafted players, red shades indicate
a decline, and white represents little or no change. This visualization
makes it easy to compare trends both across time and between schools,
revealing programs with consistent improvement, sudden drops, or highly
variable draft production. While some individual year-to-year jumps may
appear large,especially when starting from a low number of drafted
players,the overall long-term trend is more important. For example, Ole
Miss shows very few red areas and more blue or white regions, suggesting
relatively steady or improving draft representation over the 15-year
period. It is important to note that the concentration or intensity of
colors is less significant than identifying which programs display the
least amount of red, indicating greater consistency or growth.

\begin{verbatim}
## 
## =====================================
## Poisson Model for School: Alabama 
## =====================================
## 
## Call:
## glm(formula = count ~ start_year, family = "poisson", data = .x)
## 
## Coefficients:
##              Estimate Std. Error z value Pr(>|z|)  
## (Intercept) -78.70693   37.15994  -2.118   0.0342 *
## start_year    0.04017    0.01841   2.182   0.0291 *
## ---
## Signif. codes:  0 '***' 0.001 '**' 0.01 '*' 0.05 '.' 0.1 ' ' 1
## 
## (Dispersion parameter for poisson family taken to be 1)
## 
##     Null deviance: 12.9544  on 14  degrees of freedom
## Residual deviance:  8.1484  on 13  degrees of freedom
## AIC: 74.926
## 
## Number of Fisher Scoring iterations: 4
## 
## 
## =====================================
## Poisson Model for School: Arkansas 
## =====================================
## 
## Call:
## glm(formula = count ~ start_year, family = "poisson", data = .x)
## 
## Coefficients:
##             Estimate Std. Error z value Pr(>|z|)
## (Intercept) 39.20161   57.59650   0.681    0.496
## start_year  -0.01869    0.02855  -0.655    0.513
## 
## (Dispersion parameter for poisson family taken to be 1)
## 
##     Null deviance: 9.2735  on 14  degrees of freedom
## Residual deviance: 8.8438  on 13  degrees of freedom
## AIC: 62
## 
## Number of Fisher Scoring iterations: 4
## 
## 
## =====================================
## Poisson Model for School: Auburn 
## =====================================
## 
## Call:
## glm(formula = count ~ start_year, family = "poisson", data = .x)
## 
## Coefficients:
##             Estimate Std. Error z value Pr(>|z|)
## (Intercept)  5.28548   48.17515   0.110    0.913
## start_year  -0.00171    0.02387  -0.072    0.943
## 
## (Dispersion parameter for poisson family taken to be 1)
## 
##     Null deviance: 12.977  on 14  degrees of freedom
## Residual deviance: 12.972  on 13  degrees of freedom
## AIC: 71.313
## 
## Number of Fisher Scoring iterations: 4
## 
## 
## =====================================
## Poisson Model for School: Florida 
## =====================================
## 
## Call:
## glm(formula = count ~ start_year, family = "poisson", data = .x)
## 
## Coefficients:
##             Estimate Std. Error z value Pr(>|z|)
## (Intercept)  6.17915   40.05169   0.154    0.877
## start_year  -0.00197    0.01985  -0.099    0.921
## 
## (Dispersion parameter for poisson family taken to be 1)
## 
##     Null deviance: 8.9560  on 14  degrees of freedom
## Residual deviance: 8.9462  on 13  degrees of freedom
## AIC: 73.36
## 
## Number of Fisher Scoring iterations: 4
## 
## 
## =====================================
## Poisson Model for School: Georgia 
## =====================================
## 
## Call:
## glm(formula = count ~ start_year, family = "poisson", data = .x)
## 
## Coefficients:
##               Estimate Std. Error z value Pr(>|z|)    
## (Intercept) -153.05739   39.70222  -3.855 0.000116 ***
## start_year     0.07695    0.01966   3.914 9.07e-05 ***
## ---
## Signif. codes:  0 '***' 0.001 '**' 0.01 '*' 0.05 '.' 0.1 ' ' 1
## 
## (Dispersion parameter for poisson family taken to be 1)
## 
##     Null deviance: 40.686  on 14  degrees of freedom
## Residual deviance: 24.855  on 13  degrees of freedom
## AIC: 88.325
## 
## Number of Fisher Scoring iterations: 5
## 
## 
## =====================================
## Poisson Model for School: Kentucky 
## =====================================
## 
## Call:
## glm(formula = count ~ start_year, family = "poisson", data = .x)
## 
## Coefficients:
##               Estimate Std. Error z value Pr(>|z|)   
## (Intercept) -180.33008   68.94369  -2.616  0.00891 **
## start_year     0.08994    0.03413   2.635  0.00841 **
## ---
## Signif. codes:  0 '***' 0.001 '**' 0.01 '*' 0.05 '.' 0.1 ' ' 1
## 
## (Dispersion parameter for poisson family taken to be 1)
## 
##     Null deviance: 17.0752  on 12  degrees of freedom
## Residual deviance:  9.6551  on 11  degrees of freedom
## AIC: 52.171
## 
## Number of Fisher Scoring iterations: 4
## 
## 
## =====================================
## Poisson Model for School: LSU 
## =====================================
## 
## Call:
## glm(formula = count ~ start_year, family = "poisson", data = .x)
## 
## Coefficients:
##              Estimate Std. Error z value Pr(>|z|)
## (Intercept) -21.29563   39.79504  -0.535    0.593
## start_year    0.01165    0.01972   0.591    0.555
## 
## (Dispersion parameter for poisson family taken to be 1)
## 
##     Null deviance: 12.018  on 14  degrees of freedom
## Residual deviance: 11.668  on 13  degrees of freedom
## AIC: 76.161
## 
## Number of Fisher Scoring iterations: 4
## 
## 
## =====================================
## Poisson Model for School: Mississippi St 
## =====================================
## 
## Call:
## glm(formula = count ~ start_year, family = "poisson", data = .x)
## 
## Coefficients:
##              Estimate Std. Error z value Pr(>|z|)
## (Intercept) -1.439230  58.379064  -0.025     0.98
## start_year   0.001434   0.028928   0.050     0.96
## 
## (Dispersion parameter for poisson family taken to be 1)
## 
##     Null deviance: 11.482  on 13  degrees of freedom
## Residual deviance: 11.480  on 12  degrees of freedom
## AIC: 60.711
## 
## Number of Fisher Scoring iterations: 5
## 
## 
## =====================================
## Poisson Model for School: Missouri 
## =====================================
## 
## Call:
## glm(formula = count ~ start_year, family = "poisson", data = .x)
## 
## Coefficients:
##              Estimate Std. Error z value Pr(>|z|)
## (Intercept) -41.93586   60.46835  -0.694    0.488
## start_year    0.02147    0.02996   0.717    0.474
## 
## (Dispersion parameter for poisson family taken to be 1)
## 
##     Null deviance: 16.309  on 14  degrees of freedom
## Residual deviance: 15.794  on 13  degrees of freedom
## AIC: 66.646
## 
## Number of Fisher Scoring iterations: 5
## 
## 
## =====================================
## Poisson Model for School: Oklahoma 
## =====================================
## 
## Call:
## glm(formula = count ~ start_year, family = "poisson", data = .x)
## 
## Coefficients:
##              Estimate Std. Error z value Pr(>|z|)
## (Intercept) -1.568147  47.922607  -0.033    0.974
## start_year   0.001692   0.023747   0.071    0.943
## 
## (Dispersion parameter for poisson family taken to be 1)
## 
##     Null deviance: 11.685  on 14  degrees of freedom
## Residual deviance: 11.680  on 13  degrees of freedom
## AIC: 70.388
## 
## Number of Fisher Scoring iterations: 4
## 
## 
## =====================================
## Poisson Model for School: Ole Miss 
## =====================================
## 
## Call:
## glm(formula = count ~ start_year, family = "poisson", data = .x)
## 
## Coefficients:
##               Estimate Std. Error z value Pr(>|z|)    
## (Intercept) -198.40776   55.50177  -3.575 0.000350 ***
## start_year     0.09910    0.02748   3.606 0.000311 ***
## ---
## Signif. codes:  0 '***' 0.001 '**' 0.01 '*' 0.05 '.' 0.1 ' ' 1
## 
## (Dispersion parameter for poisson family taken to be 1)
## 
##     Null deviance: 28.242  on 14  degrees of freedom
## Residual deviance: 14.520  on 13  degrees of freedom
## AIC: 68.257
## 
## Number of Fisher Scoring iterations: 5
## 
## 
## =====================================
## Poisson Model for School: South Carolina 
## =====================================
## 
## Call:
## glm(formula = count ~ start_year, family = "poisson", data = .x)
## 
## Coefficients:
##              Estimate Std. Error z value Pr(>|z|)  
## (Intercept) -91.29226   50.12543  -1.821   0.0686 .
## start_year    0.04611    0.02483   1.857   0.0633 .
## ---
## Signif. codes:  0 '***' 0.001 '**' 0.01 '*' 0.05 '.' 0.1 ' ' 1
## 
## (Dispersion parameter for poisson family taken to be 1)
## 
##     Null deviance: 15.152  on 14  degrees of freedom
## Residual deviance: 11.661  on 13  degrees of freedom
## AIC: 69.112
## 
## Number of Fisher Scoring iterations: 4
## 
## 
## =====================================
## Poisson Model for School: Tennessee 
## =====================================
## 
## Call:
## glm(formula = count ~ start_year, family = "poisson", data = .x)
## 
## Coefficients:
##               Estimate Std. Error z value Pr(>|z|)  
## (Intercept) -123.64827   52.10019  -2.373   0.0176 *
## start_year     0.06211    0.02580   2.407   0.0161 *
## ---
## Signif. codes:  0 '***' 0.001 '**' 0.01 '*' 0.05 '.' 0.1 ' ' 1
## 
## (Dispersion parameter for poisson family taken to be 1)
## 
##     Null deviance: 18.002  on 14  degrees of freedom
## Residual deviance: 12.083  on 13  degrees of freedom
## AIC: 68.075
## 
## Number of Fisher Scoring iterations: 4
## 
## 
## =====================================
## Poisson Model for School: Texas 
## =====================================
## 
## Call:
## glm(formula = count ~ start_year, family = "poisson", data = .x)
## 
## Coefficients:
##               Estimate Std. Error z value Pr(>|z|)  
## (Intercept) -122.15516   49.75994  -2.455   0.0141 *
## start_year     0.06141    0.02464   2.492   0.0127 *
## ---
## Signif. codes:  0 '***' 0.001 '**' 0.01 '*' 0.05 '.' 0.1 ' ' 1
## 
## (Dispersion parameter for poisson family taken to be 1)
## 
##     Null deviance: 15.4250  on 14  degrees of freedom
## Residual deviance:  9.0827  on 13  degrees of freedom
## AIC: 67.172
## 
## Number of Fisher Scoring iterations: 4
## 
## 
## =====================================
## Poisson Model for School: Texas A&M 
## =====================================
## 
## Call:
## glm(formula = count ~ start_year, family = "poisson", data = .x)
## 
## Coefficients:
##              Estimate Std. Error z value Pr(>|z|)
## (Intercept) -65.49009   51.29694  -1.277    0.202
## start_year    0.03330    0.02541   1.310    0.190
## 
## (Dispersion parameter for poisson family taken to be 1)
## 
##     Null deviance: 10.2256  on 14  degrees of freedom
## Residual deviance:  8.4975  on 13  degrees of freedom
## AIC: 65.196
## 
## Number of Fisher Scoring iterations: 4
## 
## 
## =====================================
## Poisson Model for School: Vanderbilt 
## =====================================
## 
## Call:
## glm(formula = count ~ start_year, family = "poisson", data = .x)
## 
## Coefficients:
##              Estimate Std. Error z value Pr(>|z|)
## (Intercept) -27.54289   75.86504  -0.363    0.717
## start_year    0.01411    0.03759   0.375    0.707
## 
## (Dispersion parameter for poisson family taken to be 1)
## 
##     Null deviance: 12.220  on 14  degrees of freedom
## Residual deviance: 12.079  on 13  degrees of freedom
## AIC: 56.351
## 
## Number of Fisher Scoring iterations: 5
\end{verbatim}

\section{Analysis}\label{analysis}

For the analysis, we plan to use descriptive statistics and visual
models to track changes in the number of NFL players from each SEC
school over the three time periods. We'll look at the overall trend for
each school using a simple linear regression, which will help show
whether the number of NFL players from that school is going up or down
over time. We'll also use percentage change calculations to measure
growth rates and rank the schools based on overall gains or declines.
These methods should make it clear which schools have built a stronger
NFL presence over time. We expect that powerhouse programs like Alabama
and Georgia will show a steady growth, while schools like Vanderbilt and
Auburn may show slight growth or even a decline. For further analysis
discussing the histograms hopefully the year by year displays for each
collegiate program in the conference will further contribute to our
theory of steady growth for the elite programs while the lesser known
programs won't display such trends. We expect to find that most if not
all programs within the conference will display some form of growth
within the 2011-2025 time period and then a variation of growth and
decay during the covid year as not all programs held the same appeal
during the pandemic. With big name schools such as Alabama, Georgia and
Texas will display steady growth and the values will remain relatively
high compared to lesser known schools such as Auburn and Vanderbilt.

\section{Conclusion}\label{conclusion}

\includegraphics[trim=1cm]{FinalReport_files/figure-latex/unnamed-chunk-8-1}

Our analysis of NFL player representation from SEC schools between 2011
and 2025 reveals several major trends. First, the conference as a whole
shows a steady increase in total NFL players over time, with the
increase accelerating after 2018. Georgia, Texas, Alabama, Tennessee,
and Ole Miss show the strongest growth, each demonstrating statistically
significant positive year-to-year trends supported by their Poisson
regression coefficients. Georgia, in particular, exhibits the steepest
upward trend, with draft representation nearly doubling across the time
period.

On the other hand, schools such as Vanderbilt, Mississippi State, and
Missouri show weak or inconsistent growth, with several periods of
decline. The COVID-19 years (2020--2021) produced noticeable
irregularities in many programs due to changes in eligibility and player
decisions, but the long-term trajectories for powerhouse programs remain
strongly positive.

Overall, our results demonstrate that the SEC continues to strengthen
its NFL pipeline. The growth in talent production is not evenly
distributed, as it is concentrated in dominant programs with strong
recruiting, coaching stability, and national exposure. These findings
highlight increasing inequality within the conference and reinforce the
SEC's reputation as the premier producer of NFL talent.

\section{Contributions}\label{contributions}

Omar Sepulveda: Gathered data sets, completed data section, code used to
clean data set, code for the visuals, completed analysis section, and
wrote step by step process in README.md

Riccardo Gutierrez: Gathered data sets, completed abstract section, and
contributed to visualization and analysis section.

Naomi Menard: Completed introduction, gathered data sets, contributed to
data, visuals/code, analysis sections. Created descriptions for visuals.

\includegraphics[trim=1cm]{FinalReport_files/figure-latex/unnamed-chunk-9-1}

\end{document}
